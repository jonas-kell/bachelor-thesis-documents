This thesis tried to serve as an introduction for pure computer scientists to a comparably approachable problem in computational quantum many body physics - \emph{ground state search} with the \emph{transverse field Ising model}.
An outlook into more advanced computational methods like $VMC$ and $DMC$ was given.
This had the intent of motivating aspiring computer scientists with experiences in neural network architectures to take interest in physically motivated calculation problems as opposed to the established task like \emph{natural language processing} or \emph{image classification}.

The general concept of the specialized neural networks of the \emph{metaformer architecture} were explained.
Furthermore these networks were extended to be able to efficiently work on problems that can only be encoded in a \emph{graph} representation and not in the established square tensor format.

The \emph{inductive biases} of different neural network building blocks were presented as a motivation for why the current machine learning calculations operate the way they do.

Solving the \emph{image classification} problem with small networks, proved that the envisioned \emph{graph masked attention} was in fact working as expected and possesses the capability to outperform the full transformer by a whole complexity class if implemented properly.
At the same time, the equivalent operations to \emph{convolutions} and \emph{pooling} were introduced and tested for graph representations.

A \emph{ground state search} implementation was modified to support a range of different \emph{2-dimensional lattices}.
The search was first performed with an established \emph{CNN} architecture and validated with the \emph{numerical solution} for a problem with a small number of lattice sites.

Finally the graph metaformers were used as the \emph{NQS} for the $VMC$ ground state search.
There it could be shown, that the graph architectures can easily be used as a sufficient parameterization for solving the computational search.
And while the graph metaformers were sadly slower in solving the problem compared to the established CNN, the experiments showed them offering a very high degree of customizability due to their structure, good precision and built in resiliency against changes to the graph encoding.

The hope is, to further demonstrate their general performance benefits in representing highly complicated wavefunctions, closer to the \emph{phase transition} of quantum mechanical systems.
And while not all variations of the metaformer will be applicable to discussed computational tasks, already the thesis' small scale experiments have shown the concept to be successful.

Thus, the field of \emph{graph metaformers} and their application in \emph{quantum mechanical computational problems} offers many possibilities for further research, as the vast number of applications sadly surpasses the scope of only one bachelor thesis.
