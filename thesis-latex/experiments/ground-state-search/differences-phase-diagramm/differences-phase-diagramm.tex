The same quantum mechanical system can have different states that are described by wavefunctions of different complexity.
An example might be the behavior of a system as it approaches a specific temperature.
This could be the \emph{boiling point} of a fluid, at which the liquid turns into steam and starts behaving completely differently.
It could also be the approach to \emph{absolute zero}, that is commonly known to be responsible to induce unique behavior in quantum objects.
Depending on the nature of the system, the approach to such a point could make the wavefunction simpler or more difficult to parameterize.

Temperature is not the only parameter that could induce a transition, which is generally known as a \emph{phase transition}.
In our case the \emph{strength of the transverse electromagnetic field} can also cause such a phase transition.

The location of the transition is dictated by the ratio of the Ising parameters $J$ and $h$, $\lambda = \frac{h}{J}$, as well as the shape and dimensionality of the lattice.
The experiments in the preceding sections all used a $\lambda = \frac{h=-0.7}{J=-1} = 0.7$.
In the book \emph{Quantum Ising Phases and Transitions in Transverse Ising Models} \cite{isingBook}, the transverse field Ising phase transition is said to occur around $\lambda = $ \SIrange[]{2}{4}{} for 2D latices of square nature.

Experiments with a CNN around the region of $J = -1$, $h = -3.6$ have shown a significantly larger minimum variance of around \SI[]{0.3}[]{}. 
Compared to the experiments in the past sections this is larger by several orders of magnitude.
Bringing the $h$ parameter closer to zero, made the minimum variance drop quickly ($\approx 0.02$ for $h = -1.7$ down to falling to $< \SI[]{10e-3}[]{}$ for $h = -0.7$, where the computation was stopped after 70 steps).

The reason for operating close to the phase transition region is the increased complexity of the wavefunction.
Previous sections suggested, that the metaformer architecture might be too \glqq complex\grqq{} to represent the basic wavefunctions around $\lambda = 0.7$. 
This might be the reason, why the extremely simple CNN is outperforming the sophisticated metaformer in these examples.

It would be interesting to know, whether the metaformer can describe the complex wavefunctions for larger $\lambda$ better than e.g. a CNN.

While this might be the case, it also might not be.
During experiments on this question, for some runs a better performance was achieved. 
Other runs saw the metaformer or both models crashing due to \emph{NaN} errors after only a few steps.

As presenting a case, in that the metaformer happened to perform better than the CNN would correspond to severe cherry picking, it was decided against it.
Though the exploration of the phase transition region provides an excellent opportunity for further research in this domain.
