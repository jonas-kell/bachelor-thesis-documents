When using data for training neural networks, a consistent \emph{encoding} has to be used.
In \autoref{fig:image-classification-general} an image is transformed into a neural network input by reshaping its pixels into one column of numbers. This also works for images with multiple color channels.
A highly useful property of images is, that they can directly be encoded into \emph{tensors} depending on their format (e.g. a $224\times224$ matrix for a grayscale image or a $3\times224\times224$ tensor for an image with three color channels).

By leaving this structure in the way it already is stored in, properties like the pixel-proximity relation is conserved (while reordering into one column obviously does not preserve this information in the same way).
Network elements like \emph{convolutions} (\autoref{sec:architectures-convolution}) can use this preserved information to their advantage. 
The utilization of such kind of information is called an \emph{implicit bias} and discussed in-depth in \autoref{sec:biases}.

The canonical representation of images in memory is also the reason, why \emph{graphics} processing units (GPUs), are highly optimized to run calculation on tensors.
Because of their hardware design, operations like matrix multiplication or tensor addition can be carried out highly parallelized and efficient.

The hardware was originally designed to render graphics to the computer screen. 
It is however no longer used exclusively for that.
If a computational problem can be reduced to solving a matrix calculation, it can easily be carried out by a GPU.

Sadly not all data is generally representable in tensor form. 

A common data structure, in that many nodes exist which have relations defined between them, is called a \emph{graph}.
Examples for graph data are social media friendship relations or street data inside a navigational system.
As graphs are quite versatile however, a lot of things can be represented with them.

The novel concept is, to represent the lattice structures from \autoref{sec:theory-latticepatterns} as a graph and then apply neural network architectures, that are specifically designed to deal with that kind of data.

A big advantage is, that as graphs are more versatile as the strict tensor representation, the tensor data of an image can \emph{also} be represented as a graph. 
A square picture can for example be represented as a graph, if every pixel is treated as its own node. 
By connecting the neighboring pixels with edges, the problem can be transferred into a graph representation, without losing the information about the proximity of pixels in relation to each other.

Established operations like \emph{pooling} (\autoref{sec:architectures-perceptron-pooling}) or \emph{convolutions} (\autoref{sec:architectures-convolution}) can be defined to work on graph data. 
That will be done in \autoref{sec:architectures-graphs}. 
The advantage of the possibility to represent images in both encodings is, that the functionality of the graph operators can be verified and compared against the established implementations for tensors.
That way, one can be confident, that the calculations will also work for data that is \emph{not} representable in a tensor form.

As a graph with one node per pixel of an image would be quite large - even for low resolution images - larger chunks of the image, called a \emph{patch} (e.g. $16 \times 16$ pixel) can be treated as one unit. 
With that method, localized interactions can be computed inside a patch, while mid to far range interactions can be computed between nodes/patches, while at the same time drastically cutting down the number of nodes in the graph.

Patching has already been used extensively in \emph{vision transformer} architectures (\autoref{sec:architectures-attention}) \cite{dinoPaper} \cite{imageWorth16x16}. 
The graph-versions of these transformers are however relatively uncommon.

% SOURCE: \cite{bertPaper}
% - importance of bi-directional nature, that can look into the future (for us important, as there is not time axis in images and everything can be seen at once)
% -> decided against using that source here, because talking about image and language processing over complicates the section