This section is supposed to introduce the fundamentals of (many body) quantum physics.
The state of a quantum mechanical system is described by the wavefunction $\Psi$. 
The wavefunction encodes this state while possibly looking at a wide possible range of parameters (position in space, shape of the underlying potential, angular momentum, spin, etc.).
The goal is, to find a representation, that can encode the quantum state with large precision, while requiring as little inputs and parameters as possible. The choice of the general form used for representing the wavefunction is called the \emph{ansatz}. 
If a wrong ansatz is chosen, the quantum state cannot be encoded efficiently/at all. This will come into play in later chapters.

A wavefunction is typically written in the \emph{bra-ket notation} (see \autoref{eq:braket}). The advantage of this notation is, that it is \emph{independent} of the chosen \emph{base system} \cite[]{schwablQM}. This is used to simplify later notation.  

\begin{equation}
    \label{eq:braket}
    \Psi(x) \rightarrow \ket{\Psi} \qquad \Psi_n(x) \rightarrow \ket{n}
\end{equation}

In this case $x$ denotes an arbitrary coordinate. $\ket{\Psi}$ is an arbitrary wavefunction. $\ket{n}$ is a system of $n$ wavefunctions. Depending on the parameterization, there is a finite or infinite amount of different $\ket{n}$. In \fullref{sec:theory-ising} the base system used in this application will be introduced. In this thesis only systems with finite $n$s will be discussed.

The following properties will be important:

The representation in an other base (\autoref{eq:base-change}), where $c_n$ is a complex number that can be obtained via \autoref{eq:base-factors}.
\begin{equation}
    \label{eq:base-change}
    \ket{\Psi} = \sum\limits_{}^{n} c_n \cdot \ket{n}
\end{equation}
\begin{equation}
    \label{eq:base-factors}
    c_n = \bra{n}\ket{\Psi} = \Psi(n)
\end{equation}

For an \emph{orthogonal base} \autoref{eq:base-orthagonal} is true, for an \emph{orthonormal base} even \autoref{eq:base-orthonormal}. $\delta_{n, m}$ is called \emph{Kronecker-$\delta$} \cite{schwablQM}.

\begin{equation}
    \label{eq:base-orthagonal}
    \bra{m}\ket{n} = \begin{cases}
         = 0 &: n\neq m\\
         \neq 0 &: n = m
    \end{cases}
\end{equation}
\begin{equation}
    \label{eq:base-orthonormal}
    \bra{m}\ket{n} = \begin{cases}
         = 0 &: n\neq m\\
         = 1 &: n = m
    \end{cases} = \delta_{n, m}
\end{equation}

A wavefunction is the correct representation for a quantum system, if it satisfies the differential equation of the problem, the so called \emph{(time-independent) schrödinger equation} (\autoref{eq:schroedinger-general}), where \hamiltonian is called the \emph{hamiltonian} of the system. The hamiltonian basically \glqq encodes\grqq{} everything relevant to the problem. If must be seen as an operator, that performs a transformation of $\ket{\Psi}$, so $\hamiltonian = E$ is \emph{not} the simplified version of \autoref{eq:schroedinger-general}.

\begin{equation}
    \label{eq:schroedinger-general}
    \hamiltonian \ket{\Psi} = E \cdot \ket{\Psi}
\end{equation}

$E$ is the Energy of the system. As can bee seen here, the schrödinger equation is a linear differential equation and the correct $\ket{\Psi}$ needs to be chosen, in order for it to be fulfilled. Later this will be transformed into a matrix-vector equation. In that case, $\ket{\Psi}$ needs to be an \emph{eigenvector} to the matrix \hamiltonian.

The challenge that will be taken on in this thesis is not only to find a valid $\ket{\Psi}$, but also to find the so called \emph{ground state}. The ground state is the eigenstate with the smallest eigenvalue, speaking the smallest \emph{energy} $E = E_0$. 
This is very interesting for the physical application, because as we know from physics, a system always tries to transition into the state  with the smallest energy. A ball will not stay on a slope, but roll down into the valley, where it can sit stable. The quantum system (generally) also tries to exist in the ground state (if possible). Therefore, the shape of the ground state tells a lot about the \glqq default\grqq{} behavior of the system.