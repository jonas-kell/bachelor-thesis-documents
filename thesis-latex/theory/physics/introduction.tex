In the beginning the fundamental physical concepts will be defined. 
The goal is to introduce the physical theory in a way that explains the basics, but doesn't sacrifice the accuracy. 
By defining the used models and methods, while giving a basic insight into the more advanced notation, the hope is to introduce computer scientists with their knowledge in neural network architectures to the pending problems of many body quantum physics.

In general, all of physics is derived from fundamental differential equations. 
The classical \emph{equations of motion} for example can be derived from the fundamental differential equations $\vec{v} = \frac{\mathrm{d}\vec{r}}{\mathrm{d}t}$ and $\vec{F} = m \cdot \frac{\mathrm{d}\vec{v}}{\mathrm{d}t}$ \cite{demtroderExperimentalphysik}.
These can generally be solved (not always analytically) if the boundary conditions are known.

Quantum systems also have fundamental differential equations, the \emph{schrödinger equation} probably being the most important one. Apart from the obstacle that in quantum mechanics the boundary conditions generally cannot be measured exactly, a lot of problems can be boiled down to finding a specific solution to the schrödinger equation.