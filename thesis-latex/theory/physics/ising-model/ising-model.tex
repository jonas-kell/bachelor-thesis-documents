In this thesis, a special case of quantum system will be discussed:
The \emph{Ising model}.
The model describes multiple \emph{spins} that are locked in place. 
A spin is a very special kind of angular momentum that is inherent to quantum mechanical particles.
It helps to imagine the spin as a tiny magnet, that is locked in place, but can freely rotate to react to external/neighboring  magnetic fields. This is a very harsh oversimplification, but enough for our purposes.

The state of the \emph{many body wavefunction} can solely be described by the direction that each of the spins is pointing in. In this simplified case, only \emph{spin-$\frac{1}{2}$} particles are used. That means each spin at each position can only either be pointing \emph{up} \up or \emph{down} \down (in relation to the z-axis).

That means, that a system with $n=3$ spins (in arbitrary, but fixed position) can be represented like in \autoref{eq:spin-ising}. $\vec{s}_m$ is the direction of the spin vector, that can (in this case) be simplified to the z-axis contribution $s^z_m$ of the vector. The values can only be $\frac{1}{2}$ (\up) or $-\frac{1}{2}$ (\down).
In the given example the spins at the positions 1 and 2 point up, the one at 3 points down.

\begin{equation}
    \label{eq:spin-ising}
    \ket{\vec{s}_1, \vec{s}_2, \vec{s}_3} \rightarrow \ket{s^z_{1}, s^z_{2}, s^z_{3}} = \ket{\tfrac{1}{2}, \tfrac{1}{2}, -\tfrac{1}{2}} = \ket{\up, \up, \down}
\end{equation}

The hamiltonian for the system is defined in \autoref{eq:hamiltonian-ising}. $J_{i, j}$ is the \emph{coupling constant} between the lattice sites $i$ and $j$. $h_i$ is a measure for the external magnetic field (in x-direction) at the lattice site $i$. Often the terms each get multiplied with $-1$ (sing flip of $J$ and $h$) as per convention. This was not done in this case in order for the equation to reflect the code. The $\hat{s}^a_i$ are \emph{operators}, that measure the $s^a_i$ number of the wavefunction (possibly altering its state in the process).
$a$ in this case denotes the direction: $a \in \left\{x,y,z\right\}$.

\begin{equation}
    \label{eq:hamiltonian-ising}
    \hamiltonian = \sum\limits_{i, j} J_{i, j} \cdot \hat{s}^z_i \hat{s}^z_j + \sum\limits_{i} h_i \cdot \hat{s}^x_i
\end{equation}

\autoref{eq:hamiltonian-ising-nn} is a simplification, that assumes homogeneous interaction constants $J$ and $h$ for all sites, as well as only allowing nearest neighbors ($\langle i, j\rangle$) to interact.

\begin{equation}
    \label{eq:hamiltonian-ising-nn}
    \hamiltonian =  J \cdot \sum\limits_{\langle i, j \rangle} \hat{s}^z_i \hat{s}^z_j + h \cdot \sum\limits_{i} \hat{s}^x_i
\end{equation}

In this form, a $J < 0$ leads to so-called \emph{ferromagnetic} interaction. With a $J > 0$ the interaction is called \emph{anti-ferromagnetic}. When $h \neq 0$, the model is called \emph{transverse field Ising model} \cite{isingBook}.

The model was first solved for a 1D chain by Ernst Ising in 1924 \cite{isingFerromagnetismn}. He solved the ferromagnetic case without a transverse field (however with an additional field parallel to the z-direction). His solution was analytically derived, but already not trivial to come up with.
With the introduction of a transverse field and a more complex lattice structure than a linear chain, an analytic solution to even the most simplified hamiltonian (\autoref{eq:hamiltonian-ising-nn}) does not exist.

In order to be able to use the hamiltonian, a few rules about the interaction of the used operators and the wavefunction will be introduced. They represent only a small subset of the mathematics and focus only on the bare minimum needed to understand the problem as someone that is unfamiliar with quantum mechanics. 

The spin operators can be written in terms of the \emph{Pauli spin matrices} ($\sigma^a, a \in \left\{x,y,z\right\}$ in \autoref{eq:pauli-rules}) \cite{schwablQM}. This is only mentioned, because it reflects the implementation in the code. Note that in the code, the $\frac{1}{2}$ factors in \autoref{eq:pauli-rules} are omitted for simplification. $i$ stands for the imaginary unit.

\begin{equation}
    \label{eq:pauli-rules}
    \hat{s}^x = \frac{1}{2} \cdot \sigma^x = \frac{1}{2} \cdot \left(\begin{matrix}
        0& 1 \\
        1& 0
    \end{matrix}\right) \quad
    \hat{s}^y = \frac{1}{2} \cdot \sigma^y = \frac{1}{2} \cdot \left(\begin{matrix}
        0& -i \\
        i& 0
    \end{matrix}\right) \quad
    \hat{s}^z = \frac{1}{2} \cdot \sigma^z = \frac{1}{2} \cdot \left(\begin{matrix}
        1& 0 \\
        0& -1
    \end{matrix}\right)
\end{equation}

With the corresponding Pauli representation of the spin vectors in \autoref{eq:pauli-vectors}, the interaction of the operators can be directly computed with standard matrix multiplication. The important cases are calculated in \autoref{eq:pauli-transformation}.

\begin{equation}
    \label{eq:pauli-vectors}
    \ket{\up} = \left(\begin{matrix}
        1\\0
    \end{matrix}\right)
    \qquad
    \ket{\down} = \left(\begin{matrix}
        0\\1
    \end{matrix}\right)
\end{equation}


\begin{equation}
    \label{eq:pauli-transformation}
    \sigma^z \ket{\up} = +1 \cdot \ket{\up} \quad
    \sigma^z \ket{\down} = -1 \cdot \ket{\down} \quad
    \sigma^x \ket{\up} = +1 \cdot \ket{\down} \quad
    \sigma^x \ket{\down} = +1 \cdot \ket{\up}
\end{equation}

The last important piece of information is, that operators denoted by indices only interact with the spin at the corresponding lattice site. Three examples are given in \autoref{eq:pauli-many-body}. Multiple operators acting on one state can be computed by only evaluating the effect of the rightmost operator according to the mentioned rules and then repeating until finished. Complex numbers can always be commutatively swapped with operators and bra-kets.

\begin{equation}
    \label{eq:pauli-many-body}
    \sigma^z_2 \ket{\down, \up, \down} = +1 \cdot \ket{\down, \up, \down} \quad
    \sigma^z_1 \ket{\down, \up, \down} = -1 \cdot \ket{\down, \up, \down} \quad
    \sigma^x_3 \ket{\down, \down, \down} = +1 \cdot \ket{\down, \down, \up}
\end{equation}  